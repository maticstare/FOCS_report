
\documentclass{article}

\usepackage{amsmath}
\usepackage{amssymb}
\usepackage{graphicx}
\usepackage{caption}
\usepackage{subcaption}
\usepackage{cite}
\usepackage{hyperref}
\usepackage{geometry}
\usepackage{fancyhdr}
\usepackage{setspace}
\usepackage{enumitem}
\usepackage{lipsum}
\usepackage{float}
\usepackage{enumitem}
\usepackage{amsfonts}
\usepackage{tikz}

%\graphicspath{ {./graphs/} }

% Page layout
\geometry{top=1in, bottom=1in, left=1.5in, right=1.5in}
\pagestyle{fancy}
\fancyhf{}
%\rhead{Matic Stare, Martin Starič, Veljko Dudić, Don Mohar, Matej Kovač, Rene Mlinarič and Žan Štuhec}
\lhead{FOCS: Annual IEEE Symposium on Foundations of Computer Science 2023}
\cfoot{\thepage}
\renewcommand{\headrulewidth}{0.4pt}
\renewcommand{\footrulewidth}{0.4pt}

% Title
\title{FOCS: Annual IEEE Symposium on Foundations of Computer Science 2023}
\author{
  Stare, Matic\\
  \and
  Starič, Martin\\
  \and
  Dudić, Veljko\\
  \and
  Mohar, Don
  \and
  Kovač, Matej
  \and
  Mlinarič, Rene
  \and
  Štuhec, Žan
}
\date{\today}

\begin{document}

\maketitle

% Table of Contents
\tableofcontents
\newpage

% Sections
\section{Introduction}
64th IEEE Symposium on Foundations of Computer Science (FOCS) 2023 je uradno ime konference ki je potekala od 6. do 9. novembra 2023 v mestu Santa Cruz (Kalifornija, Združene države Amerike). Njene teme so bile: algoritmi in podatkovne strukture, kriptografija, računska kompleksnost, teorija računalniškega učenja, iskanje informacij in baze podatkov. Oddanih prispevkov je bilo 421, a so jih sprejeli le 142. Vseh 142 je bilo tudi predstavljeno.

\section{A Randomized Algorithm for Single-Source Shortest Path on Undirected Real-Weighted Graphs}

Problem najkrajše poti je eden izmed najbolj znanih problemov v teoriji grafov. Cilj je, da najdemo najkrajšo pot med dvema vozliščema na podlagi uteženih grafov. Za reševanje tovrstnih problemov se je v preteklosti najbolj uveljavil Dijkstrin algoritem, ki deluje s časovno zahtevnostjo $O(n^2)$, kjer je $n$ število vozlišč. To mejo so kasneje izboljšali s Fibonaccijevo kopico na $O(n \log n + m)$, kjer je $m$ število povezav. V tem članku~\cite{duan2023randomized} so se avtorji osredotočili na problem najkrajše poti iz enega vira (SSSP) na uteženih grafih. Zasnovali so nov algoritem, ki deluje s časovno zahtevnostjo $O(m \sqrt{\log n * \log \log n})$, vendar je pogojen z verjetnostjo. To pomeni, da je časovna zahtevnost algoritma pričakovana in ne zahtevnost v najslabšem primeru. Poleg tega je vneseni graf $G$ potrebno spremeniti v graf s konstantnimi stopnjami~\cite{frederickson1983data}, kar pomeni da ima nov dobljeni graf $G'$ $O(m)$ vozlišč in povezav.

Do nedavnega je bilo ozko grlo pri iskanju SSSP vrsta s prednostjo. Ideja izboljšave je, da v vrsto vnesemo manjše število vozlišč. To dosežemo s tehniko \textit{Bundle Construction}, ki deluje na sledeč način:
\begin{itemize}
  \item Iz prvotne množice vozlišč $V$ naključno izberemo podmnožico $R_1 \subseteq V$, pri čemer je vsako vozlišče vsebovano v $R_1$ z verjetnostjo $\frac{1}{k}$. V množici $R_1$ se nahaja tudi vir $s$.
  \item Na vsakem $v \notin R_1$ poženemo Dijkstrin algoritem z začetkom v $v$.
  Vsako vozlišče, ki izločimo iz kopice dodamo v urejeno množico $V_{extract}^{(v)}$. Ko naletimo na $u \in R_1$, oziroma je moč množice $|V_{extract}^{(v)}| > k \log k$, se Dijkstrin algoritem z začetkom v $v$ zaključi.
  \item Če je bila moč množice $|V_{extract}^{(v)}| > k \log k$, potem dodamo vozlišče $v$ v množico $R_2$.
  \item Nastavimo $R := R_1 \cup R_2$ in prvo vozlišče v $V_{extract}^{(v)} \cap R =: b(v)$ 
  \item Z zgornjim rezultatom izračunamo sledeče: $Bundle(u)$ za vsak $u \in R$, $Ball(v)$ za vsak $v \in V \backslash R$ in $dist(v,u)$ za vsak $u \in Ball(v) \cup \{b(v)\}$.
\end{itemize}

Ker je pričakovana moč množice $|V_{extract}^{(v)}| = O(k)$ za vsak $v$ in je v grafu $G'$ $O(m)$ vozlišč, algoritem terja časovno zahtevnost $O(mk \log k)$.

Sledi algoritem \textit{Bundle Dijkstra}, ki na osnovi pridobljene množice paketov posodablja razdalje vseh vozlišč do vira. Gre za relaksacijo vozlišč $v \in V \backslash R$.

\textit{Bundle Dijkstra} - Ustvarimo tabelo, kjer bomo hranili razdalje vozlišč $v$ do vozlišča $s$, in v Fibonaccijevo kopico vstavimo vsa vozlišča iz $R$.
\begin{enumerate}
  \item
  Ko iz kopice izločimo vozlišče $u$, posodobimo tabelo tako, da za vsako vozlišče $v$, ki je zapakirano vozlišču $u$ najdemo točno razdaljo. To storimo tako, da posodobimo $d(v)$ glede na $d(u)$, vozlišča v $Ball(v)$ in sosednja vozlišča vozlišču $v$ in $Ball(v)$.
  \item
  Ko je vsako vozlišče $x \in Bundle(u)$ posodobimo še sosednja vozlišča $y \in N(x)$ in vozlišča, ki so znotraj $Ball(y)$.
  \item
  Ko posodobimo vozlišče, ki ni v množici $R$, posodobimo še njegovo zapakirano vozlišče $b(v)$ z $d(v) + dist(v,b(v))$.
\end{enumerate}
Ker je v algoritmu vstavljenih le $|R|$ vozlišč, je časovna kompleksnost izločanja elementov iz kopice enaka $O(|R|\log n)$. Poleg izločanja elementov iz kopice, k časovni kompleksnosti pripomore še zanka skozi vsa vozlišča ${v} \in V \backslash R$, vendar ker se v algoritmu preverjajo tudi vozlišča znotraj $Ball(v)$ je končna časovna zahtevnost \textit{Bundle Dijkstra} $O(\frac{m}{k}\log n + mk)$.

Tako je skupna časovna zahtevnost celotnega zasnovanega algoritma enaka $O(\frac{m}{k}\log n + mk \log k)$, ki jo je možno minimizirati z izbranim $k = \frac{\log n}{\log \log n}$.
\section{Polynomial-Time Pseudodeterministic Construction of Primes}
Gat in Goldwasser sta postavila vprašanje ali je mogoče izračunati praštevila psevdoderministično v polinomskem času. Naključnostni algoritem je psevdodeterminističen, če izračuna fiksno kanonično rešitev za iskalni problem z visoko verjetnostjo. Članek predstavlja rešitev na ta problem v neskončno-pogostem režimu. Predstavlja brezpogojni polinomski naključnostni algoritem B, tako da za neskončno mnogo vrednosti n, $B(1^n)$ izračuna kanonično n-bitno praštevilo $p_n$ z visoko verjetnostjo. Splošneje za vsako gosto lastnost Q nizov, ki jo je mogoče določiti v polinomskem času, obstaja neskončno-pogosta psevdodeterministična polinomska konstrukcija nizov, ki zadovoljijo Q. Pristop presega subeksponentno časovno konstrukcijo, predlagano s strani Oliveire in Santhanama. Vključuje različne inovativne koncepte, kot je nova tehnika zagona za psevdodeterministične konstrukcije, ter kvalitativno izboljšanje okvira enakomerne trdnosti-naključnosti, ki sta ga predstavila Chen in Tell, s pomočjo variante generatorja Shaltiel-Umans.

\subsection{Rezultati}
Glavni izrek določa pogoje, pod katerimi probabilistični algoritem s polinomskim časom, skupaj s sekvenco nizov iz določenega jezika, zagotavlja specifične verjetnostne izide. Z uporabo tega izreka na jeziku praštevil članek predstavi naključni algoritem s polinomskim časom, ki z visoko verjetnostjo zanesljivo proizvaja praštevila za neskončno mnogo vhodnih dolžin, prekašajoč prejšnje metode subexponentne časovne zahtevnosti.

\textbf{Izrek 1 (Neskončno pogosto polinomski čas Psevdodeterminističnih konstrukcij).} Naj bo \( Q \subseteq \{0, 1\}^* \) jezik z naslednjimi lastnostmi:
\begin{itemize}
    \item \textbf{(Gostota.)} obstaja konstanta \( \rho \geq 1 \) tako, da za vsak \( n \in \mathbb{N}_{\geq 1} \), \( Q_n \triangleq Q \cap \{0, 1\}^n \) ustreza \( |Q_n| \geq n^{-\rho} \cdot 2^n \); in
    \item \textbf{(Preprostost.)} obstaja determinističen polinomski čas algoritem \( A_Q \), ki odloča, ali vhod \( x \in \{0, 1\}^* \) pripada \( Q \).
\end{itemize}
Nato obstaja verjetnostni polinomski čas algoritem \( B \) in zaporedje \( \{x_n\}_{n \in \mathbb{N}_{\geq 1}} \) nizov \( n \)-bitnih nizov v \( Q \) tako, da veljajo naslednji pogoji:
\begin{enumerate}
    \item Pri vsaki vhodni dolžini \( n \in \mathbb{N}_{\geq 1} \), \( \Pr[B(1^n) \notin \{x_n, \perp\}] \leq 2^{-n} \).
    \item Na neskončno mnogih vhodnih dolžinah \( n \in \mathbb{N}_{\geq 1} \), \( \Pr[B(1^n) = x_n] \geq 1 - 2^{-n} \).
\end{enumerate}


\subsection*{Posledice}
\begin{itemize}
    \item Obstaja verjetnostni polinomski algoritem B, ki za neskončno mnogo vrednosti $n$ izpiše kanonično n-bitno praštevilo $p_n$ z visoko verjetnostjo. To je dokaz moči teorije kompleksnosti v reševanju problemov, ki se zdijo biti o številski teoriji.
    \item Obstaja konstanta $c \geq 1$, tako da za $t(n) = n^c$ velja naslednje: Za vsak $m \geq 1$ obstaja $n > m$ in $n$-bitno praštevilo $p_n$, pri čemer je $rK^{t}(p_n) \leq \log(n) + O(1)$. Drugače povedano, obstaja neskončno mnogo praštevil, ki imajo zelo kratke učinkovite opise.
\end{itemize}
\section{Proof of the Clustered Hadwiger Conjecture}
Preden avtorji dokažejo gručasto Hadwigerjevo domnevo, opišejo naslednje pojme.\\

Polni grafi so grafi, v katerih je vsako vozlišče povezano z vsakim drugim vozliščem. Poln graf z $n$ vozlišči označimo $K_n$.\\                                                           
Minorji grafa $G$ so podgrafi, ki jih dobimo z brisanjem povezav in vozlišč ter krčenjem povezav $G$.\\
Barvanje grafa je preslikava, ki vsakemu vozlišču dodeli barvo.\\
Graf je $k$-obarvljiv, ko za njegovo obarvanje porabimo največ $k$ barv.\\
Graf je pravilno obarvan, ko ima vsak sosednji par vozlišč med seboj različno barvo.\\
Kromatično število grafa $G$ je najmanjše tako število $k$, da je $G$ pravilno obarvan in $k$-obarvljiv.\\
\\
Hadwigerjeva domneva:\\
Naj bo $K_h$ poln graf na $h$ vozliščih.\\
Hadwiger je domneval, da je vsak graf brez minorja $K_h$ pravilno $(h - 1)$-obarvljiv.\\
\\
Monokromatična komponenta obarvanega grafa $G$ so med seboj povezana vozlišča enake barve.\\
Gručevje je število vozlišč največje monokromatične komponente obarvanega grafa $G$.\\
Gručasto monokromatično število množice grafov je najmanjše tako število $k$, za katerega obstaja število $c$, da je vsak graf iz množice $k$-obarvljiv z gručevjem $c$.\\
\\
Gručasta Hadwigerjeva domneva(v nadaljevanju izrek 1) trdi, da je vsak graf brez minorja $K_h$ $(h - 1)$-obarvljiv z gručevjem največ neke funkcije $f(h)$.\\
\\
Avtorji nadaljujejo z gručastim kromatičnim številom množice grafov brez minorja $K_s,t$, kjer je $K_s,t$ poln bipartitni graf z deli velikosti $t>=s>=1$.\\
Izrek2:\\
Vsak graf brez minorja $K_s,t$ je $(s+1)$-obarvljiv s gručevjem največ neke funkcije $f(s,t)$.\\
S pomočjo izreka 2 dokažejo, da je omenjeno gručasto kromatično število $s+1$.\\

Avtorji nato izreka 1 in 2 posplošijo na izrek 4:\\
Vsak graf brez minorja $J_s,t$ je $(s+1)$-obarvljiv z gručevjem največ $f(s, t)$.\\

Temu iz dokazanih izrekov Liu-a in Wood-a ter definicije za 'apex' graf, ki ob odvzemu največ 1 vozlišča postane planarni graf, sledi izrek 7:\\
Za katerikoli celi števili $t >= s >= 3$ in katerokoli temeljišče grafa $X$, vsak graf $K_{s,t}$ brez podgrafov in minorjev v $X$ je $(s + 1)$-obarvljiv s clusteringom največ $f(s, t, X)$.\\

Dokazna metoda za izreka 4 in 7 uporablja napredne tehnike teorije grafov za reševanje kompleksnih problemov. Ključna sestavina teh dokazov je uporaba 'teorije strukture produkta grafov' skupaj z izrekom Robertsona in Seymourja, ki obravnava strukturo grafovskih minorjev. To omogoča preoblikovanje planarnih grafov v enostavnejše grafe z omejeno drevesno širino, kar olajša analizo in manipulacijo z njimi.

Pristop je sicer omejen na družine grafov, ki omogočajo močan produkt grafa. Zato se zanašamo na izrek Robertsona in Seymourja, ki razčleni grafe brez določenega vzorca na manjše dele, imenovane "torzi". Te "torzi" so sestavljene iz vdelanega podgrafa na površini, dopolnjenega z vrtinci in temeljišči s prosto okolico, podobno kot drevo kličnih vsot.

Pri barvanju določenih vrst grafov sledimo strategiji barvanja "torzij" enega za drugim, pri čemer pazimo, da ne prekrijemo že obarvanih delov y novo barvo. Teorem 4 nam zagotavlja, da lahko uporabimo največ s+1 barv, vendar to ni vedno izvedljivo, zato moramo "torze", ki jih ni mogoče pobarvati v s+1 barvah, združiti v zavese. Za barvanje celotnega grafa preprosto pobarvamo vsako zaveso zaporedno, začevši s tisto, ki vsebuje koreninski "torzo". Pri tem posvečamo pozornost določenim vrstam točk v grafu, imenovanim "temeljišča".

Za dodatno poenostavitev grafa uporabljamo tudi metode, kot sta particioniranje in plastenje, da ohranimo omejenost velikosti presečišč vsakega dela in vsake plasti. Namesto ustvarjanja omejenega grafa količnika drevesne širine, povzdignemo zaveso, da oblikujemo manjši G↑ (G minor) z omejeno drevesno širino. S tem zagotovimo, da G↑ ostane minor od G, s čimer dopolnimo dokazno strategijo za izrek 4.

Naši rezultati ponujajo konstruktivne dokaze in algoritme polinomskega časa $O(n^c)$, neodvisne od izključenega minorja ali podgrafa, kar omogoča učinkovito obvladovanje K2,t-podgrafov. Izreka 1 in 2 zagotavljata optimalne meje za gručasto kromatsko število neminornih grafov Kh in Ks,t, poleg ključno razvitih definicij in orodij, kot so dekompozicije zavese, dvignjene zavese in kontrakcije. Pričakuje se, da bodo ta orodja uporabna v različnih kontekstih zaradi njihove učinkovitosti pri nadzoru K2,t-podgrafov.


\bibliography{references}
\bibliographystyle{plain}

\end{document}